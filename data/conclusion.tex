% !TeX root = ../sustechthesis-example.tex

\begin{conclusion}
本研究基于动力学稳定性判据,构建了一种适用于细长杆稳定性分析的数值判定方法,为蛇形条带的稳定性研究提供了有效的数值分析工具。随后,基于基尔霍夫弹性杆理论,将蛇形结构建模为一个多段边界值问题。利用数值延拓工具包COCO求解该边值问题,并进行分岔分析,详细研究了蛇形结构在后屈曲状态下的分岔行为以及多稳态特性。最后,在对蛇形结构详细的分岔分析的基础上,通过优化结构几何参数实现了对各类分岔点位置和顺序的精确调控,进而有效控制了结构的后屈曲行为。本文的主要研究结果如下:
\begin{enumerate}
	\item 基于动力学结构稳定性判据,构建一种适用于细长杆结构稳定性分析的数值判定方法,并通过多个算例验证了该稳定性测试方法的可靠性。
	\item 通过对蛇形结构的分岔分析,揭示了单单元蛇形结构的前两阶屈曲模态随结构高度变化而发生交换的数学机理——多重特征值分岔。在屈曲模态交换过程中,前两阶屈曲模态对应的两个分岔点首先相互靠近,随后融合,最终分离。在分岔点融合时,对应着多重特征值分岔;而在分岔点分离时,多重特征值分岔点会分裂出二次分岔点,该二次分岔点的出现促使前两阶分岔支的稳定性发生交换。另外本文详细研究了蛇形结构的多稳态特性,通过数值计算得到了蛇形结构在拉伸载荷下的多个稳态构形,所得构形与实验所得构形相吻合。另外,阐明了蛇形结构的可逆对称性,并给出各对称构形之间解的关系。
	\item 通过优化方法,对蛇形结构的后屈曲行为进行了有效调控。在双单元蛇形结构中,通过调控不同条带段的高度,实现了前两阶屈曲模态的交换。进一步对优化后的结构进行分岔分析,揭示了多单元结构中屈曲模态交换同样是由多重特征值分岔引起的。这一发现再次强调了多重特征值分岔对模态交换的重要性。
\end{enumerate}

由于研究时间和本人水平所限,关于蛇形结构的后屈曲行为的研究在本文中尚不充分,下一步可展开的工作为:
\begin{enumerate}
	\item 本文中对结构多稳态的研究是基于实验的方法,但该方法无法保证获得所有可能的稳态解。因此,未来可以发展一种数值方法,系统地求解该结构在某一固定载荷下的所有稳态解,以更全面地揭示其力学行为。
	\item 本文主要对蛇形结构进行了力学分析并提出了一种有效的优化方法,但未深入探索该结构在实际工程中的应用潜力。因此,未来可以进一步研究该结构在软体机器人、超材料等领域的应用。

\end{enumerate}
\end{conclusion}
